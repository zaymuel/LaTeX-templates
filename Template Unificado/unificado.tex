% Copyright (c) 2021 zaymuel under MIT License
% For more info, https://github.com/zaymuel/LaTeX-templates

\documentclass[12pt, a4paper]{article}

% mudar a cada documento
\newcommand{\NOMEROTEIRO}{Prova X / Trabalho X}
\newcommand{\NOMEPROF}{NOMEPROF}
\newcommand{\NOMEMATERIA}{NOMEMATERIA}
\newcommand{\TURMA}{TURMA}
\newcommand{\SEMESTRE}{202X/X}
\newcommand{\DEPARTAMENTO}{Instituto de Física / Departamento de Matemática}

\newcommand{\MEUNOME}{MEUNOME}
\newcommand{\MINHAMATRICULA}{MINHAMATRICULA}
\newcommand{\MINHAUNIVERSIDADE}{Universidade de Brasília}

\usepackage{geometry}
 \geometry{                     % margens ABNT
 a4paper, left=3cm, top=3cm, right=2cm, bottom=2cm
 }

% peguei o seguinte bloco de alguem (não sei quem), e modifiquei.

% Pacotes básicos 
% ----------------------------------------------------------
\usepackage{lmodern}		    	% Usa a fonte Latin Modern	
%--------------------------------------
% Encoding - essential!
\usepackage[T1]{fontenc}		  % Seleção de códigos de fonte.
\usepackage[utf8]{inputenc}		% Codificação do documento (conversão automática dos acentos)
%--------------------------------------
% Regras de Hifenização
\usepackage{hyphenat}
\hyphenation{mate-mática recu-perar}
%--------------------------------------
\usepackage{indentfirst}	       	% Indenta o primeiro parágrafo de cada seção.
\usepackage{color}		           	% Controle das cores
\usepackage{graphicx}		         	% Inclusão de gráficos
\graphicspath{ {./images/} }      % define diretório raiz para imagens,
% então é só colocar o nome do arquivo em \includegraphics{nomedoarquivo}
\usepackage{microtype} 		       	% para melhorias de justificação
\usepackage{ifthen}		          	% para montar condicionais
\usepackage[brazil]{babel}	     	% para utilizar termos em português
\usepackage[autostyle]{csquotes}  % fazer as aspas aparecerem corretamente
  \MakeOuterQuote{"}              % poder usar aspas duplas ("") em vez de (``'')
\usepackage[final]{pdfpages}      % para incluir páginas de arquivos PDF
\usepackage{lipsum}				        % para geração de lorem ipsum
\usepackage{float} 		            % Para a figura ficar na posição correta
\usepackage{caption}              % configura legenda 
    \captionsetup{font=small}	    % tamanho da fonte 10pt na legenda
\usepackage{footmisc}             % precisa ir antes de hyperref, senao referencias a rodapes quebram
\usepackage[colorlinks=true]{hyperref} % links coloridos

% Pacotes que permitem as mais comuns "coisas matemáticas"
\usepackage{amsmath,amsthm,amssymb}
\usepackage{bm}

\usepackage[arrowdel]{physics}    % coisas físicas, diferenciais (\dd) e derivadas (\dv)
\usepackage{siunitx}    % unidades padrão do SI

\usepackage{cancel}     % cancelamento de termos na equação

% \usepackage{multirow}   % celulas em tabelas com varias linhas
% \usepackage{enumitem}
% \usepackage{pdflscape}  % escolher páginas para irem em paisagem (landscape)

% \usepackage{minted}      % incluir código-fonte
%   \usemintedstyle{borland}

% Environments for `answer' and `resolution'
% https://tex.stackexchange.com/questions/211136/environment-solution-environment-for-exercises-different-than-proof-environme
\newenvironment{resposta}
  {\renewcommand\qedsymbol{$\blacksquare$}\begin{proof}[Resposta, conforme a subsequente resolução]$ $\par\nobreak\ignorespaces}
  {\end{proof}}
\newenvironment{resolucao}
  {\renewcommand\qedsymbol{$\blacksquare$}\begin{proof}[Resolução]$ $\par\nobreak\ignorespaces}
  {\end{proof}}
  
% OPCIONAL: abaixo, para não precisar escrever "Questão"
% https://tex.stackexchange.com/questions/245089/how-to-change-the-section-title-and-its-arrangement-in-a-latex-document
\renewcommand{\thesection}{Questão \arabic{section}} % Questão 1, Questão 2, etc
\usepackage{titlesec}
\titleformat{\section}
{\normalfont\Large\bfseries}{\thesection}{1em}{}
\renewcommand{\thesubsection}{(\alph{subsection})}  % (a), (b), etc
\usepackage{titlesec}
\titleformat{\subsection}
{\normalfont\large\bfseries}{\thesubsection}{1em}{}


% usando vetores unitários: \uvec{i}, \uvec{j}
\DeclareRobustCommand{\uvec}[1]{{%
  \ifcsname uvec#1\endcsname
     \csname uvec#1\endcsname
  \else
    \bm{\hat{\mathbf{#1}}}%
  \fi
}}

\usepackage[backend=biber, style=numeric, 
sorting=none  % sort pela ordem de ocorrência
]{biblatex}   % usa o biblatex, ferramenta + moderna para a bibliografia
    \addbibresource{references.bib}     % puxa as entries de bibliografia desse arquivo


\title{\NOMEROTEIRO \\
    {\Large \NOMEMATERIA \ -- Turma \TURMA \ -- \SEMESTRE} \\
    {\large \emph{Professor} \NOMEPROF \\
    \DEPARTAMENTO \\
     \MINHAUNIVERSIDADE}}
\author{
\text{\MEUNOME} -- \MINHAMATRICULA
}
\date{\today}


\begin{document}
% titulo, nome, data, etc
% \maketitle

\begin{titlepage}
  \centering

  % logo UnB
  \begin{figure}[htbp] % usar [H] para inserir EXATAMENTE AQUI
    \vspace*{\fill} % se quiser encher a pagina
    \centering
    \noindent
    \makebox[\textwidth]{
      % \includegraphics[scale=1]{unb_logo_comprido.pdf}
      \includegraphics[scale=2]{unb_logo.pdf}    % University Logo
    }
  \end{figure}

  \vspace*{1 cm}

  \textsc{\huge \MINHAUNIVERSIDADE}\\[0.5 cm]	% University Name
  \textsc{\Large \DEPARTAMENTO}\\[0.8 cm]	% Department Name
  \rule{\linewidth}{0.2 mm} \\[0.4 cm]
  {\huge \bfseries \NOMEROTEIRO}\\
  \rule{\linewidth}{0.2 mm} \\[1.2 cm]
  \textsc{\LARGE \NOMEMATERIA \ -- {\Large Turma} \TURMA \ -- \SEMESTRE}\\[1 cm]			% Course Code, Turma

  \begin{minipage}{0.35\textwidth}
    \begin{flushleft} \large
      \emph{Professor:}\\
      \NOMEPROF
    \end{flushleft}
  \end{minipage}~
  \begin{minipage}{0.6\textwidth}
    \begin{flushright} \large
      \text{\emph{Estudante:}} \\
      \text{\MEUNOME} -- \MINHAMATRICULA \\
    \end{flushright}
  \end{minipage}\\

  \vspace{1 cm}
  {\large \today}

  \tableofcontents

  \vspace*{\fill}

\end{titlepage}


\newcommand{\EXEMPLOITEMa}
{Encontre a solução da equação diferencial dada por:
  % \begin{equation}
  %   x=x
  %   \label{q1}
  % \end{equation}
}

% [pages=`n`] para página específica,
% [pages=`n-m`] para range das páginas `n` a `m`,
% [pages=`-`] para todas as páginas.
% \includepdf[pages=1]{images/F2e de.pdf}
\section{} % Questão 1
Considerações gerais e instruções da questão. Ex: ``Considere $$I = \iiint\limits_E \dd{x_1} \dd{x_2} \dd{x_3}$$...''

\subsection{} % Letra a
Item (a) da questão. \EXEMPLOITEMa
% \paragraph{(a)} Alternative display of an item on the question. \EXEMPLOITEMa
Ex: ``Esboço do sólido  $E'$ em coordenadas esféricas $\rho, \theta, \phi$ que irá corresponder a $E$.''

% resolução do item (a)
\begin{resposta}
  \begin{align*}
    y=y
  \end{align*}
\end{resposta}

\begin{resolucao}
  \begin{align}
    z=z
    \label{q2}
  \end{align}
  Integrando ambos os lados, temos:
  \begin{align*}
    \int{ \dv{}{t} \left[\mu(t)y(t)\right] \dd{t} } & = \int{ [\mu(t) \cdot t] \dd{t} }
                                                    &                                   & \text{pelo Teorema Fundamental do Cálculo:} \\
    \dots
  \end{align*}
\end{resolucao}

% \begin{landscape}
% Visão geral da tabela usada:
% \end{landscape}

% \begin{alignat*}{2}
%     0 &\le \theta & &\le 2 \pi \\
%     0 &\le \phi   & &\le \frac{\pi}{4} \\
%     0 &\le \rho   & &\le \cos{\phi}
% \end{alignat*}

%% using images, `ctrl' + `/' to uncomment
%% escolher entre 1 dos seguintes modos:
% \begin{figure}[htbp]    % onde ficar melhor
% \begin{figure}[H]       % EXATAMENTE AQUI
% \vspace*{\fill} % se quiser encher a pagina
% \centering\noindent
% % definir escala arbitraria ou largura do texto
% \makebox[\textwidth]{
%   \includegraphics[scale=0.5]{images/1.png}
%   \includegraphics[width=0.9\textwidth]{images/1.png}
% }
% \caption{Brasil \protect\cite[58]{snis}}
% \label{fig:esboco-1a}
% \vspace*{\fill}
% \end{figure}


\end{document}